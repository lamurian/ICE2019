\documentclass[]{article}
\usepackage{lmodern}
\usepackage{amssymb,amsmath}
\usepackage{ifxetex,ifluatex}
\usepackage{fixltx2e} % provides \textsubscript
\ifnum 0\ifxetex 1\fi\ifluatex 1\fi=0 % if pdftex
  \usepackage[T1]{fontenc}
  \usepackage[utf8]{inputenc}
\else % if luatex or xelatex
  \ifxetex
    \usepackage{mathspec}
  \else
    \usepackage{fontspec}
  \fi
  \defaultfontfeatures{Ligatures=TeX,Scale=MatchLowercase}
\fi
% use upquote if available, for straight quotes in verbatim environments
\IfFileExists{upquote.sty}{\usepackage{upquote}}{}
% use microtype if available
\IfFileExists{microtype.sty}{%
\usepackage{microtype}
\UseMicrotypeSet[protrusion]{basicmath} % disable protrusion for tt fonts
}{}
\usepackage[margin=1in]{geometry}
\usepackage{hyperref}
\hypersetup{unicode=true,
            pdftitle={Graph Database on Medical Research Data for Integrated Life Science Research},
            pdfauthor={Aly Lamuri; Jonathan Midlando Adithama Purba; Adrian Reynaldo Sudirman; Randy Sarayar},
            pdfkeywords={graph database, medical informatics},
            pdfborder={0 0 0},
            breaklinks=true}
\urlstyle{same}  % don't use monospace font for urls
\usepackage{graphicx,grffile}
\makeatletter
\def\maxwidth{\ifdim\Gin@nat@width>\linewidth\linewidth\else\Gin@nat@width\fi}
\def\maxheight{\ifdim\Gin@nat@height>\textheight\textheight\else\Gin@nat@height\fi}
\makeatother
% Scale images if necessary, so that they will not overflow the page
% margins by default, and it is still possible to overwrite the defaults
% using explicit options in \includegraphics[width, height, ...]{}
\setkeys{Gin}{width=\maxwidth,height=\maxheight,keepaspectratio}
\IfFileExists{parskip.sty}{%
\usepackage{parskip}
}{% else
\setlength{\parindent}{0pt}
\setlength{\parskip}{6pt plus 2pt minus 1pt}
}
\setlength{\emergencystretch}{3em}  % prevent overfull lines
\providecommand{\tightlist}{%
  \setlength{\itemsep}{0pt}\setlength{\parskip}{0pt}}
\setcounter{secnumdepth}{5}
% Redefines (sub)paragraphs to behave more like sections
\ifx\paragraph\undefined\else
\let\oldparagraph\paragraph
\renewcommand{\paragraph}[1]{\oldparagraph{#1}\mbox{}}
\fi
\ifx\subparagraph\undefined\else
\let\oldsubparagraph\subparagraph
\renewcommand{\subparagraph}[1]{\oldsubparagraph{#1}\mbox{}}
\fi

%%% Use protect on footnotes to avoid problems with footnotes in titles
\let\rmarkdownfootnote\footnote%
\def\footnote{\protect\rmarkdownfootnote}

%%% Change title format to be more compact
\usepackage{titling}

% Create subtitle command for use in maketitle
\providecommand{\subtitle}[1]{
  \posttitle{
    \begin{center}\large#1\end{center}
    }
}

\setlength{\droptitle}{-2em}

  \title{Graph Database on Medical Research Data for Integrated Life Science
Research}
    \pretitle{\vspace{\droptitle}\centering\huge}
  \posttitle{\par}
    \author{Aly Lamuri \\ Jonathan Midlando Adithama Purba \\ Adrian Reynaldo Sudirman \\ Randy Sarayar}
    \preauthor{\centering\large\emph}
  \postauthor{\par}
    \date{}
    \predate{}\postdate{}
  
\usepackage{float}
\floatplacement{figure}{H}

\begin{document}
\maketitle
\begin{abstract}
Indonesia is having an increasing surge of published scientific articles
during recent years. In medical science, published articles greatly vary
from both pre-clinical and clinical studies where each study possesses
different methodological approach and hypothetical premises. However,
some articles do not include a rigorous documentation as to make it
reproducible. Moreover, the lack of centralized database further impedes
researcher from reanalyzing previous findings and integrating them with
the new study. This paper delineates such an issue by constructing a
graph database to centralize and integrate clinical research data.
Database is constructed using \texttt{Neo4j} and \texttt{cypher}
querying language populated with 5,000 medical records generated by
\texttt{synthea} program. Our database able to answer queries requiring
complex relationship while minimizing the amount of database hits. As a
concluding remark, graph database is quite performant to solve data
integration and centralization issue faced by life science research
institutes.
\end{abstract}

\hypertarget{introduction}{%
\section{Introduction}\label{introduction}}

Scientific publication in Indonesia underwent manifold increases within
the past decades. Reported by Maula, Fuad and Utarini (2018), numbers of
published article on dengue-related subject increased 13 times in 2017
as compared to 2007. Such an increase also followed by \emph{h-index}
improvement, resulting in Indonesia placed as the 5\textsuperscript{th}
most scientifically productive ASEAN country in investigating
dengue-related topic (Maula, Fuad and Utarini, 2018). Another
bibliometric analysis investigated by Sarwar and Hassan (2015) also
enlisted Indonesia within 11 most scientifically productive Islamic
countries. However, these articles often neither robustly elaborate the
methodological procedure nor provide obtained data for reanalysis, two
factors contributing to reproducibility and credibility issue in
scientific publication (Pashler and Wagenmakers, 2012; Stark, 2018;
Resnik and Shamoo, 2016). Besides enabling preprint access
(Oakden-Rayner, Beam and Palmer, 2018) and thorough documentation on
methodology, data availability is also a crucial component for
reproducibility in science (Peng, 2015). Therefore, we proposed
utilizing graph database to integrate research findings in life
science-related fields.

\hypertarget{graph-database}{%
\subsection{Graph Database}\label{graph-database}}

Data management system should appropriately consider interoperability
and scalability which enable data storing, indexing and retrieving.
Databases aggregate integrated object in a structure defined by its
metadata. The presence of metadata implies a self-defined property of
the database, whereas in relational database management system (RDBMS)
such definition included within its particular schema (Berg, Seymour and
Goel, 2012). During the development of RDBMS, emerging is the need to
quickly retrieve the data through syntactically and logically feasible
manner, therefore inducing the conceptual design of \texttt{SQL}, a
structured querying language. However, with data being stored in a
multi-tabular layout, relational database (RDB) faced massive
disadvantages in handling highly-connected data. Hence the development
of schema-less database initiated by NoSQL (Berg, Seymour and Goel,
2012; Fabregat \emph{et al.}, 2018), with graph database being one of
its variants (Oussous \emph{et al.}, 2015).

Graph database is more performant in storing data with intricate
relationships, e.g protein interactions or chemical reaction pathways,
as compared to its RDB counterparts (Fabregat \emph{et al.}, 2018).
\texttt{Neo4j} is a graph database platform developed in \texttt{java}
and compliant towards ACID system (Atomicity, Consistency, Isolation,
Durability) (Oussous \emph{et al.}, 2015). As a native graph database,
\texttt{Neo4j} shall store data as explicitly defined relationships in a
schema-less management system. Therefore, \texttt{Neo4j} treats database
querying as a graph traversing process. This redeeming feature of graph
database in general enables higher performance and flexibility in
storing the data. \texttt{Neo4j} employs \texttt{cypher} as a querying
language to define patterns on traversing the relationship graph.
Furthermore, ASCII-Art syntax of \texttt{cypher} enables a more
intuitive querying process. Such uniquely written language and
ACID-compliant platform could become a two-fold advantages to use
\texttt{Neo4j} in delivering graph database management system.

\hypertarget{medical-informatics}{%
\subsection{Medical Informatics}\label{medical-informatics}}

Information in life science-related fields often possess an intelligible
relationship of causative nature. Many of such information may present
as a connection between one entity to another. Interractome, reactome
and connectome are common examples we may find in currently emerging
basic science research. In translational research paradigm, some
interests highlighted the importance of genetic and proteomic
interaction network. Meanwhile in clinical settings, we may also want to
consider patient-doctor-institution as separate yet related entities.
Therefore, the nature of data in medicine is actually a close
resemblance of entity-relationship data. Indubitably, we shall consider
applying graph database as an alternative to RDB to store life
science-related research data.

\hypertarget{method}{%
\section{Method}\label{method}}

This study utilized \texttt{Neo4j} as a platform to create graph
database with \texttt{cypher} as the querying language. Data used in
this study are generated from \texttt{synthea} program, producing 5,000
medical records in json-based FHIR (Fast Healthcare Interoperability
Resources) which directly converted into \texttt{*.csv} format. As shown
in figure \ref{fig:garuda.schema}, we treated each entity as vertex and
underlying relationship as an edge connecting two vertices. We first
design constraints for unique input and indices for redundant vertices.
To prevent random access memory (RAM) bottleneck, we enabled periodic
commit for each 500 inputs, which especially beneficial when dealing
with numerous entries. Afterwards, we load \texttt{*.csv} file generated
by \texttt{synthea} as a query object and finally set the entity and
relationship.

\begin{figure}

{\centering \includegraphics{_fig/garuda-schema} 

}

\caption{\label{fig:garuda.schema} Schematic representation on graph
database for medical records}\label{fig:fig:garuda.schema}
\end{figure}

\hypertarget{result}{%
\section{Result}\label{result}}

Constructed database is able to return answer to queries requiring with
complex relationship. We demonstrated the database capability by
querying following syntax:

\begin{verbatim}
// List patients with hypertension
match (p:Patient) -[:ATTENDED_AN]-> (e:Encounter)
match (:Diagnoses {Name:'Hypertension'}) <-[:HAS_DIAGNOSES]-
    (e) <-[:PROVIDED_AN]- (o:Organization)
return p, e, o
;
\end{verbatim}

In this query, we asked a specific information of patients with
hypertension who attended a certain institution without any time
constraint. This query returned a graph object representing network of
patient, institution and the encounter. To challenge more complicated
relationships, we also performed following query:

\begin{verbatim}
// List diagnoses in Ludlow City
match (p:Patient) -[:ATTENDED_AN]->
    (e:Encounter) <-[:PROVIDED_AN]- 
    (o:Organization) -[:LOCATED_IN]-> (g:GeoLoc)
match (d:Diagnoses) <-[r:HAS_DIAGNOSES]- (e:Encounter)
where g.Name = 'LUDLOW'
return p.Name as Patient,
    d.Name as Diagnoses,
    o.Name as Institution,
    g.Name as City,
    r.Date as Date 
;
\end{verbatim}

This query returned a table representing list of diagnoses constrained
within Ludlow City. Figure \ref{fig:dbhits1} and \ref{fig:dbhits2}
depicted profile of database hits from both queries.

\begin{figure}

{\centering \includegraphics{_fig/query-plan-1} 

}

\caption{\label{fig:dbhits1} Database hits on the first query}\label{fig:fig:dbhits1}
\end{figure}

\begin{figure}

{\centering \includegraphics{_fig/query-plan-2} 

}

\caption{\label{fig:dbhits2} Database hits on the secoond query}\label{fig:fig:dbhits2}
\end{figure}

\begin{itemize}
\tightlist
\item
  Show query example
\item
  Show how many db-hits on a query
\item
  Show figures describing query processing (require \texttt{Neo4j}
  browser)
\end{itemize}

\hypertarget{discussion}{%
\section{Discussion}\label{discussion}}

Our study demonstrates graph database as a potential platform to store
life science research data. Previous studies emphasizes on graph
database credibility on storing interconnected data, where graph
database pattern query on such data may outperform RDB (Medhi and
Baruah, 2017; Fabregat \emph{et al.}, 2018; Mathew and Kumar, 2014).
However, on other cases requiring analytical query, RDB outperformed
graph database, wherein their study Hölsch, Schmidt and Grossniklaus
(2017) argued \texttt{Neo4j} became less performant due to less advanced
disk and buffer management compared to RDB.

\begin{itemize}
\tightlist
\item
  Body: graph db as a possible solution, considering the intricate
  relationships of medical information
\item
  Premises on how graph db can be further utilized, maintained and
  developed
\item
  Conclude the usability of graph db
\end{itemize}

\hypertarget{conclusion}{%
\section{Conclusion}\label{conclusion}}

Graph database is quite performant to integrate medical health record
generated for 5,000 subjects using \texttt{synthea} program.

\hypertarget{reference}{%
\section*{Reference}\label{reference}}
\addcontentsline{toc}{section}{Reference}

\clearpage

\hypertarget{refs}{}
\leavevmode\hypertarget{ref-Berg2012}{}%
Berg, K.L., Seymour, T. and Goel, R. (2012) History of databases.
\emph{International Journal of Management \& Information Systems
(IJMIS)} {[}online{]}. 17 (1), pp. 29. Available from:
\url{https://doi.org/10.19030/ijmis.v17i1.7587}doi:\href{https://doi.org/10.19030/ijmis.v17i1.7587}{10.19030/ijmis.v17i1.7587}.

\leavevmode\hypertarget{ref-Fabregat2018}{}%
Fabregat, A., Korninger, F., Viteri, G., Sidiropoulos, K., Marin-Garcia,
P., Ping, P., Wu, G., Stein, L., D'Eustachio, P. and Hermjakob, H.
(2018) Reactome graph database: Efficient access to complex pathway data
Timothée Poisot (ed.). \emph{PLOS Computational Biology} {[}online{]}.
14 (1), pp. e1005968. Available from:
\url{https://doi.org/10.1371/journal.pcbi.1005968}doi:\href{https://doi.org/10.1371/journal.pcbi.1005968}{10.1371/journal.pcbi.1005968}.

\leavevmode\hypertarget{ref-Holsch2017}{}%
Hölsch, J., Schmidt, T. and Grossniklaus, M. (2017) On the performance
of analytical and pattern matching graph queries in neo4j and a
relational database. In: Yannis Ioannidis (ed.). \emph{Proceedings of
the workshops of the edbt/icdt 2017 joint conference} CEUR workshop
proceedings {[}online{]}. 2017 Aachen: CEUR-WS.org. Available from:
\url{http://ceur-ws.org/Vol-1810/GraphQ_paper_01.pdf}.

\leavevmode\hypertarget{ref-Mathew2014}{}%
Mathew, A.B. and Kumar, S. (2014) An efficient index based query
handling model for neo4j. \emph{IJCST}. 3 (2), pp. 12--18.

\leavevmode\hypertarget{ref-Maula2018}{}%
Maula, A.W., Fuad, A. and Utarini, A. (2018) Ten-years trend of dengue
research in indonesia and south-east asian countries: A bibliometric
analysis. \emph{Global Health Action} {[}online{]}. 11 (1), pp. 1504398.
Available from:
\url{https://doi.org/10.1080/16549716.2018.1504398}doi:\href{https://doi.org/10.1080/16549716.2018.1504398}{10.1080/16549716.2018.1504398}.

\leavevmode\hypertarget{ref-Medhi2017}{}%
Medhi, S. and Baruah, H. (2017) Relational database and graph database:
A comparative analysis. \emph{Journal of Process Management. New
Technologies} {[}online{]}. 5 (2), pp. 1--9. Available from:
\url{https://doi.org/10.5937/jouproman5-13553}doi:\href{https://doi.org/10.5937/jouproman5-13553}{10.5937/jouproman5-13553}.

\leavevmode\hypertarget{ref-OakdenRayner2018}{}%
Oakden-Rayner, L., Beam, A.L. and Palmer, L.J. (2018) Medical journals
should embrace preprints to address the reproducibility crisis.
\emph{International Journal of Epidemiology} {[}online{]}. 47 (5), pp.
1363--1365. Available from:
\url{https://doi.org/10.1093/ije/dyy105}doi:\href{https://doi.org/10.1093/ije/dyy105}{10.1093/ije/dyy105}.

\leavevmode\hypertarget{ref-Oussous2015}{}%
Oussous, A., Benjelloun, F.-Z., Lahcen, A.A. and Belfkih, S. (2015)
Comparison and classification of nosql databases for big data. In:
\emph{Proceedings of international conference on big data, cloud and
applications}. 2015

\leavevmode\hypertarget{ref-Pashler2012}{}%
Pashler, H. and Wagenmakers, E. (2012) Editors' introduction to the
special section on replicability in psychological science.
\emph{Perspectives on Psychological Science} {[}online{]}. 7 (6), pp.
528--530. Available from:
\url{https://doi.org/10.1177/1745691612465253}doi:\href{https://doi.org/10.1177/1745691612465253}{10.1177/1745691612465253}.

\leavevmode\hypertarget{ref-Peng2015}{}%
Peng, R. (2015) The reproducibility crisis in science: A statistical
counterattack. \emph{Significance} {[}online{]}. 12 (3), pp. 30--32.
Available from:
\url{https://doi.org/10.1111/j.1740-9713.2015.00827.x}doi:\href{https://doi.org/10.1111/j.1740-9713.2015.00827.x}{10.1111/j.1740-9713.2015.00827.x}.

\leavevmode\hypertarget{ref-Resnik2016}{}%
Resnik, D.B. and Shamoo, A.E. (2016) Reproducibility and research
integrity. \emph{Accountability in Research} {[}online{]}. 24 (2), pp.
116--123. Available from:
\url{https://doi.org/10.1080/08989621.2016.1257387}doi:\href{https://doi.org/10.1080/08989621.2016.1257387}{10.1080/08989621.2016.1257387}.

\leavevmode\hypertarget{ref-Sarwar2015}{}%
Sarwar, R. and Hassan, S.-U. (2015) A bibliometric assessment of
scientific productivity and international collaboration of the islamic
world in science and technology (s\&T) areas. \emph{Scientometrics}
{[}online{]}. 105 (2), pp. 1059--1077. Available from:
\url{https://doi.org/10.1007/s11192-015-1718-z}doi:\href{https://doi.org/10.1007/s11192-015-1718-z}{10.1007/s11192-015-1718-z}.

\leavevmode\hypertarget{ref-Stark2018}{}%
Stark, P.B. (2018) Before reproducibility must come preproducibility.
\emph{Nature} {[}online{]}. 557 (7707), pp. 613--613. Available from:
\url{https://doi.org/10.1038/d41586-018-05256-0}doi:\href{https://doi.org/10.1038/d41586-018-05256-0}{10.1038/d41586-018-05256-0}.


\end{document}
